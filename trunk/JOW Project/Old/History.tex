\documentclass[a4paper,twocolumn]{article}
\title{\textbf{Turn Based Strategy} \\ A History}
\author{Richar Archer \\ 23076552}
\pagestyle{headings}
\begin{document}

\maketitle
\tableofcontents

\newpage
\section{Introduction}
I apoligise in advance for getting a tad carried away. This particular timeline I am almost afraid to say is very personal as I have been with it since the days of the Commodore 64. I am going to write this document on a very personal level as opposed to factual as its purpose is to establish what made these games great for me and my friends, so I might make a game that will satisfy my fellow gamers. \\
I have found the only structure I can adhere to, without rambling is an alphabetic breakdown of all the games I have played, and a few I haven't until this week (once I found out about them I made a point of playing them). The only exception to this is an old game on the Commodore 64 which will forever stay with me and was my Starting point for TBS (Turn Based Strategy). I kick myself for not being able to remember the name.

\section{Unknown Beginnings}
It arrived at x-mass as part of my brother's presents but like all games in our house it was all of ours. Most games on the C64 had great graphics and sound, considering, but this game was different. Everything was tri colour: red, white and black. The graphics where poor and sound none existent but all 64k was stuffed with content. The game played out as a squad based strategy game where you cleared out dungeons.
Everything was turn based and the units each had their own strategies and the AI was impressively difficult. We never finished it but it was my starting point in this genre.
\begin{itemize}
\item Game play over Graphics for TBS.
\end{itemize}

\section{Advanced Wars I and II}
Designed for the game boy advanced this title was released by Intelligent Systems recently. It plays out extremely well considering it's a handheld game, no mouse, and even then it's brilliant. The units move on what I call the move action system, simply put you get one move action and attack action with every unit. The game is on a grid and a small combat engine takes place for each engagement. The combat engine is pretty but can be distracting, fortunately it can be disabled.\\
The Commanding Officer system (CO) means with each officer your army changes, some COs have good movement rates with land units but poor attacks. My favorite CO capture rate is improved but her tanks are weak. The other cool thing about this game is the more you play the more content unlocks and the more you want to play. Unfortunately the sequel is more an expansion than a improvement.\\
\begin{itemize}
\item CO system adds more options
\item Powers are good
\item Cartoon graphics, clear and precise
\item Unlock able Content is good.
\end{itemize}

\section{Heroes Games}
\subsection{King's Bounty}
The first moving armies game of its sort. The concept is simple you have a hero and he collects an army and runs around a map fighting monsters until you find the King's Bounty hence the name. The heroes are limited to the Knight, Paladin, the Sorcerer or the Barbarian. This choice does not render special influence on the course of the game. It was released in 1990 by New World Computing (founded in 1984 by Jon Van Caneghem). The game was later released on the Sega as well. I must confess I didn't play it until this week and well it does not compare to its modern counter-parts I was seriously impressed. You can defiantly see the roots of Heroes released many years later.

\subsection{Heroes of Might and Magic}
Oh the hours upon hours lost to this game� I regret none of them. I joined the franchise at the second game, the one most consider the pinnacle of the series. I tend to disagree and think the third is. The general formula is very similar to King's Bounty but I didn't know it at the time, for me it was a first. Each hero was different each city was filled with fantastical monsters and pretty graphics.\\
Again the simple cartoon graphics triumph in this genre and serve their purpose. The cities interface was always beautiful because they made the cities like living paintings, which after a few hundred hours of game play you could most probably draw blindfolded but that doesn't draw away from their splendour. My biggest yes point on this game is the huge range available to you as a player, it allows for countless replayability.\\
The flaw I must say comes in multiplayer; I was one of the few I know who enjoyed it. Much like chess by email you made your move and waited while your human opponent moved. As long as alt tab is there this game play works, but it is flawed.\\
The game based on the Might and Magic RPG, many hours lost there as well, and the series has stuck to a well worn formula, through all three versions and many expansions. Again the fans couldn't get enough game content.\\
The fourth instalment wasn't a disappointment adding many new features and content but the cartoonist graphics was discarded for more life-like graphics and the new features made the game cumbersome, most of them were stolen from rival titles such as Disciples and Age of Wonders.

\subsection{Age of Wonders}
Okay I've only manage to play the sequel of this one but to be honest it is really riding the Heroes of Might and Magic wave pretty thinly. That is not to say it isn't a brilliant game, its difficult and fun with many features, facets and gadgets that HOMM lacks, at least until HOMM IV stole some, like the quick resolution. Ultimately it is the same game with twists however.\\
The focus on research is very nice and a great improvement on the HOMM formula. The graphics while pleasing lack that fantastical element and clarity. They went the more realistic route and have paid dearly for it.\\
The variance between races and teams is very pleasing if a bit heavy ended, most of the base units are the same and the differences only bloom near the end of the power scale. Which is understandable for balance but one wishes they took more risks in the team formulas.\\
The games tactical formula is well thought out in general but one gaping flaw is the games dependency on the random element. High level heroes and powerful units can sometimes fail to hit or damage measly units repeatedly. The diplomacy is also frightfully random. \\

\begin{itemize}
\item Colourful clean graphics
\item Hotkeys are good
\item Variety and Flair in teams improves replayability
\item Multiplayer can suffer from down time
\item A good formula works time and time again
\item Flavour information is cool
\item Over dependency on random factors detracts from strategy
\item Content loads of content
\end{itemize}

\section{Empire Building}
\subsection{Colonisation}
Personally I have to say Sid Meier is a genius and this is most probably one of his best works. Few know it but they all loved it. More tactical than strategic than its cousin Civilisation the game spun around independence. You pretty much started with a single ship and built up the new world.\\
The trading was well handled and the game overall played very well. The combat was quick resolution which I didn't like as it tossed too much power to the numbers and not enough towards your brain but it still heavily incorporated terrain and other effects. Fortunately there were no flashy sequences which made you go growl in anguish.\\
The graphics and interface is clean cut and very colourful. The shortcut keys are everywhere and veteran players are thankful for them. The end rating system is frustratingly brilliant as it keeps you going because having a school named after you just isn't good enough damn it all.\\
The town building is pretty, and the nice clean lists and user friendly interface make a quite complex and difficult game inviting even when your a new player. The flavour text, and endless titbits are highly satisfying.\\
The highly addictive game got you renaming all your towns and cities cause the standards names weren't 'yours' and the more ummm dedicated, or possessed fans *cough* *cough*, changed the text files to customise the ship names ect... but not me no never. I just taught others how to do it I swear, all those lies about me retyping almost every line of game dialogue are false ask my lawyer. Needless to say I still have the game installed and my PC is never without it.

\subsection{Civilisation}
Next to UFO and HOMM this is one of the three giants of turn based strategy. I have played all the originals, everything after CivNET was just lame to be honest and while I must say I enjoyed all of the games immensly I didn't enjoy them as much as I did the other games on this list. Civilisation II was great it was just a pity that it was released concurrently with CivNET.\\
The city building and empire building was ingenious and the game play was smooth, combat was too simple for my liking but to be honest the catch of the game for me is it wasn't at all fantastical. It followed the paths of history into not so radical futures. The graphics reflected the dull subject matter. The reading was factual and dull.\\
The entire Civilisation games where fun but to be honest they never grabbed me as much as Colonisation or the later Sid Meier masterpiece Alpha Centari. The game was predictable and when it comes down to it the tilt, zing, wow, or polish factor just wasn't there for me.

\subsection{Alpha Centari}
Once upon a time there was a company called Microprose, they made brilliant games no doubt about it true pioneers. Then one day Sid Meier left and the company became a corporate regurgitation device for old ideas. I will later talk about Bullfrog and other companies which suffered from the Death of the Gaming Gods.\\
Alpha Centari is a pet project of Sid Meier which fought against several minor court battles to stay free from Microprose's oppressive legal department but it was a gem to behold. I honestly haven't bought many games but I am proud to own this one. It is a box of treats, I mention these because it is all too common to open a box today and find a cd and slim boring manual.\\
This box came with a giant poster, one side pretty graphics and the other side was the complex research tree. The manual was thick and full of what I call fluffy bits, the things fans want details, trivia, and stories. The game is fuelled with brilliant short cut scenes and loads of dialogue with each item or discovery which is almost ninety percent fluffy.\\
The story is bold and the graphics colourful. The races each have different strengths and weaknesses, and there are various unique win strategies. The planet is alive and an obstacle in itself. The unit workshops which allow you to design units using your new technologies is amazing awesome, even if the prototype rule makes Spartans kind of overpowering in combat games.\\
The story is unique and you find yourself drawn into a tale of a foreign planet and a new future for the human race. You find yourself researching things, or using government strategies which aren't the best stats choice but you just want to follow an ideal because the story is so compelling. Your almost guaranteed to play at least twice to get an idea of the different outcomes.\\
Once again the creative genius shows what his legacy's true sequel and continuation is, not another rehashed game but a true extension once again pushing the envelope.

\begin{itemize}
\item Colourful graphics and Fantastical settings are good.
\item Fluffy bits, must have fluffy bits for geeks and fans
\item Polish and dangerous ideas, conventional games die fast
\item Involving stories are great
\item Customisation is good
\item Different teams, not just different names
\end{itemize}

\section{Table Top}
\subsection{War Hammer 40k}
While not a PC game you have to recognise the table top roots. Lets face it back in the old days before PC gaming pen and paper games such as dungeons and dragons where the call of the geek. Even today we still role-play and do table top games because they involve elements that the computer games have yet to capture. Computer games are still gladly drawing inspiration from these games, Dawn of War, Vampire Bloodlines, Neverwinter Nights, Mechwarrior are just a very small tip of the large iceberg which is paper inspired games.\\
War Hammer plays out on a large table about 2 by 1 meters long. The settings are fantasy and futuristic, for the most part, however many other options are available. The game plays out like well a war game. Open terrain and turn based moves are a must because real-time doesn't work in mechanics.\\
Die rolls determine the outcomes of things using reference tables, a fairly simple process. The things that players love about the game that has yet to be translated is the building and painting of your army. The army point system lets you balance out army sizes, always being able to tell an armies relative strength. That is not to say all 5000 point armies are equal, a badly obstructed army with units that don't complement each other will obviously be weaker but for the most part the balance holds.\\
The paint designs and models are colour and fantastical for the most part. The players take great pride in the personalised feel and unique qualities of their army. Often spending almost a year painting an army. This game is highly complex and I don't have enough time to go into details but I have drawn a lot of inspiration from this game.

\subsection{Confrontation}
Confrontation is a recent addition to the tabletop world. If Warhammer and its partners are Starcraft like games then Confrontation is UFO type games. Your average squad has between three and eight models. Exceptions exist, like the executioner who is a one model army and the goblins who tend to be numerous but for most parts its a small game.\\
The game is cheaper than Warhammer but the real reason I have drawn so much from this game is its one of the first skirmish games. A lot of the game mechanics can be converted over nicely for our game. The system is simple but capable of many things. Once we add a detailed hidden level the system will truly bloom.\\
The game is played on almost any terrain. In fact most of our impromptu games are played on a messy table or floor as the rules incorporate mountains (books), towers (coffee cups), rivers (cloth) and many other elements (other random junk we are too lazy to take off the table) it makes for an interesting game.

\subsection{Dawn of War}
Well not strictly turn based it draws from Warhammer and I want to talk about it. It was a real opportunity for the computer game industry to make a brilliant turn based strategy game but they opted for real time strategy, why is the question.\\
After many debates on the topic and much reading the only conclusion that experts (random friends) and me can come to is that they were afraid of failing. Simple put that the table top game is already the turn based version and they would have had to best the table top game to impress players. Also the real time element allows them to show off graphics very nicely. And well I am not normally one for graphics being able too zoom into a Mech and watching it pick up an orc in one hand and toasting him with a flame thrower just amuses me to no end.\\
\begin{itemize}
\item Skirmish is good, more involving
\item Swimming, climbing and neat stuff is great
\item Dynamic Terrain
\item Gobliiiiiiiiiins RULE
\item Little details
\item Players love having something special
\item Hmmmm Flame Throwers
\end{itemize}

\section{Final Fantasy}
Okay our game isn't going to resemble FF series but their is a lot of stuff our game can learn from this turn based series. First of all the games are just WOW!!! For the unenlightened fools who haven't lost 800+ hours of your lives to FF series (I'm not kidding in my commune there are three of us.) The FF series is special, first of all no games are linked with the exception of FF 10 and FF 10-2.\\
Secondly the FF series was started and is still run by a gaming god. He looked at gaming back in the day and saw it stagnating and decided he would make on game before leaving the industry, and he wouldn't care if the fans liked it, and thus it was his Final Fantasy, thus the name. The game was a MASSIVE success and proved to him that great things could still be done with games.\\
Since then the FF series has been one of the big envelope pushes of gaming. The pity is that only FF 7 and FF 8 ever came to PC. The game revolves around stories, epic tales three of which have brought me to tears more than once. The characters are so real and tangible that you feel like you are part of the endlessly dynamic worlds.\\
The other reason that the games are so bloody addictive is take FF 10; the story can be completed in fifty hours, which is much longer than most games which average 20 to 30 hours. in truth however the game takes over a hundred hours to unlock all the secret and max out your character. Such is typical of those Japanese *censored*, gods I love them but really unlocking some of those secrets are almost impossible and takes weeks of effort.\\
The other thing that has become better and better with each game is the hidden mechanics. Again take FF10, every ability or spell in the game has a long animation but the more you use an ability the quicker it becomes to the point the animation is almost non-existent. This is good because it stops the player thinking that he has seen all of this before. Not to mention that the summons of the earlier games have become infamous in their drawn out nature.\\
The other hidden mechanics are the emotional stats. Things like depending on how many battles Tidus fights with Lulu or how many potions you give to Yuna ect determines who throws the ball to Tidus in his ultimate move. Also your overdrive modes and other secrets are unlocked by doing something 1000 times or not doing something and so on.\\
Another thing the FMV and graphics are so pretty you just can't take your eyes off them, in fact FF FMVs in some cases are better than Blizzards, but you didn't hear me say that.

\subsection{Final Fantasy Tactics}
Very different from the other games but not any less brilliant. A skirmish game which says you PC people have UFO we Gameboy people have FFT. The game is endlessly long, which is a good thing. The story as always is brilliant and involving. The core elements are well though out. The job structure and levelling systems are smooth. The intricate game devices work.\\
What is interesting about FFT is its not original, a small group made Tactics Ogre, a highly unsuccessful flop. The game was almost identical except for the background on which it too place and the polish. Firstly Ogre Tactics was overly complex and the interface confusing. The intricate elemental devices confused even long time strategy players; the intro was long and inescapable. The story was good but not up to the epic standards of Square Cinc.\\
In fact the Tactics story proves some vital elements. Firstly the established name of Square got it a lot of attention. Secondly they made the graphics and world more colourful and fantastical. Smoothed out the playability and simplified the visible system, many complex underlying elements were added as always with Square games. The polish was added and the game story totally replaced. Ultimately the game mechanics and the basic game was the same. The difference Ogre Tactics was a massive flop and Final Fantasy Tactics was a brilliant success. Though the lack of a mouse is felt.\\
\begin{itemize}
\item Packaging does count
\item Pretty colourful graphics are a big plus
\item An involving stories keep a player emotionally involved
\item Real characters  make for a real experience
\item Simplicity with hidden depths is always the best approach
\item Unlockable content is good
\end{itemize}

\section{Squad Games}
\subsection{Laser Squad}
A small game most people don't remember but it was the true starting point of squad games. It lacked polish, clean graphics and a decent story but all the ideas where there. Julian Gollop was the creator and while it wasn't his first game, Rebelstar Raiders was a poorly distributed title which most people will give you blank stares when you mention it. In fact most people give blank stares when you mention Laser Squad or even Xcom, we should shoot them all. Oh well in 1988 the game came out on a load of systems, not the PC (it wasn't a games platform yet). The game uses turn based strategy and is very similar to the Xcom series with a few exceptions. It lacked story, fluff and used a emulated 3d type look. I've only ever played the Laser Squad Gold edition which is an ancient title which was pretty much just Laser Squad on PC and is a truly rare find. It never lived up to its later child for me Xcom will always be the successful one of the series. Lords of Chaos was Julian's next title before he moved onto UFO, his most successful game ever.

\subsection{UFO}
My favourite game of all time. Sadly I have to ashamedly admit I missed the first game and only spotted the game when Terror from the Deep came out in my primary school years and quickly stole my heart and soul, it has yet to return them. Since then Xcom: UFO Defence and Xcom: Terror from the Deep have yet to leave my PC, except for a hort period while I search for an XP version. This is the main inspiration for my game ideas and UFO has been called by many the greatest computer game of all time.\\
You had base building, you had research with loads of fluffy bits, you had planes, aliens and cool guns. The graphics were clean and clear. Admittedly TFTD is more an expansion than a sequel but we didn't complain. The game story visuals, and cheesy intro are all part of a brilliant package. The ability to rename your troopers and watch them devolp. From panicky little under powered squirts who get confused, scared, mind-controlled, or killed to uber powerful soldiers of doom who dispense justice all the way back to mars with Plasma guns and Jetpack armour from hells infernos and the ability to mind control the aliens the tables really do turn a full one eighty.\\
The fluffy bits and smooth user friendly system make the game the in-depth masterpiece it is today. Sadly it didn't have a multiplayer option, the fan base which is still highly active coded a hack tool to make it multiplayer however called UFO 2000. The genre died out for a while after Xcom: Apocalypse destroyed the credibility of the game by trying to make it real time. Things got worse from there as Xcom interceptor came out, a space fighter simulator which dragged the Xcom name through the mud further. It's important to note Julian Gollop left the team after TFTD.\\
Early last year the genre saw the rebirth with UFO, a separate title made by a new team which took all the brilliant elements of Xcom and brought them back into the game. The game took out base development and eliminated the crisp clean feel of the game and ultimately didn't achieve as much success as it should have sadly, also requiring a stupidly fast pc.\\
The Indy community however won't let UFO die, off the top of my head I can name almost ten clones in development, the most impressive at the moment appears to be UFO: AI a clone which ironically uses the Quake 2 engine. You have to see it to believe it. The project shows promise and shouldn't be taken out of the running just yet. Oh Xcom means eXtraterrestrial COunter Measures just by the way for those confused fools who haven't played the game.

\subsection{Silent Storm}
Silent Storm 2 came out last year and to be honest it is a great game which is brilliantly executed but to be honest its Xcom in World War II. A setting I am sick and tired of personally. It lacks the clean cut and cartoon graphics of other titles, opting for the more realistic look. The game plays well, the performance is good and the mechanics are smooth. Little can be said against it, I am currently playing through the last mission and thoroughly enjoying them. Why then isn't the game all over the shelves gaming accolades. The truth is sadly the publisher just isn't pushing the game enough.

\subsection{Laser Squad Nemesis}
So what is the Turn Based master Jullian Gollop doing at the moment, well its a small Indy project which is doing very well. Its called Laser Squad Nemesis and is a return to his roots, small Indy devolpment. Mostly written as an online game, but it does have a single player campaign and over 500 maps to play. The game has online leagues and uses new game mechanics involved issuing orders then watching ten seconds of combat before issuing new orders. Picture it as intelligent turn based strategy. The online leagues are well and alive. The games graphics are clean, and the game runs in a small window mode which is very friendly to people who work and play at the same time.
\begin{itemize}
\item Renaming Squad Members good
\item Cool weapons
\item Cool fluffy research
\item Fantastical stories always work better
\item Online play is wanted by the fans
\item Indy projects are cool
\end{itemize}

\onecolumn
\section{The Gods of Gaming}
I quickly want to talk about the saddest trend in gaming over the last few years, and that is the death of the Gods. Back in the day when making a game took a bunch of friends in a garage we saw a few genius who had the story telling knack rise as gaming Gods. Like great authors, directors, playwrights and artists they stood above their peers. There is no argument some people have 'it' and these people most certainly did it for the love of games.\\
Names jump out at you, Sid Meier, Jullian Gollop, John Romero, Peter Monolynx, John Carmack are all names that a certain generation of gamers remember. No genre was hit more by the death of gaming idols than turn based strategy. Luckily it looks like the trend is reversing. Also turn based strategy is in many Indy gaming projects and upcoming game line ups. I believe that once again Turn based strategy will not only see the light of day but bask in the spotlight.

\section{Conclusion}
I love turn based games, its a pity that many think them to be a dead genre but in truth I believe the genre has been biding its time and in the next three years we shall see its rebirth. Or to quote some random punk, 'Give me two lengths of metal a large storm and I SHALL MAKE IT LIVE!'.

\section{Sources}
Personal Expierence and many wasted hours.\\
http://www.gamespot.com/gba/strategy/advancewars/review.html \\
http://www.gamespot.com/pc/strategy/ageofwonders2thewt/review.html \\
http://www.the-underdogs.org/game.php?name=CivNet \\
http://www.civfanatics.com/civ1.shtml \\
http://pc.ign.com/articles/403/403490p1.html \\
http://www.gamespot.com/pc/strategy/fallouttacticsbos \\
http://www.gamespot.com/ps/strategy/finalfantasytactics/review.html \\
http://lpp.freelords.org/about.php \\
http://www.gamespy.com/articles/493/493525p1.html \\
http://www.zone.ee/kingsbounty/ \\
http://takegame.com/others/htm/kingsbounty.htm \\
http://www.freedownloadscenter.com/Games/Strategy\_ Games/Laser\_ Squad\_ Nemesis.html \\
http://l-o-c.sourceforge.net/ \\
http://www.the-underdogs.org/game.php?id=687 \\
http://www.ibiblio.org/GameBytes/issue21/greviews/momrev.html \\
http://www.cdmag.com/strategy\_ vault/max\_ review/page1.html \\
http://www.gamespot.com/pc/strategy/max/review.html \\
http://www.gamesfirst.com/reviews/ericq/mmbn3white/mmbn3white.htm \\
http://www.silentstorm\- online.com/content.php?lang=en\& rid=598 \\
http://www.ufoai.net/ \\
http://www.atariage.com/software\_ page.html?SoftwareLabelID=593 \\
http://www.rolemaker.dk/documents/Requirements/BA\_ GameReference/Strategy/Warlords2.htm \\
http://www.wesnoth.org/ \\
http://www.geocities.com/Area51/Rampart/5504/history.html \\
http://www.xcomufo.com/gameinfo.html

\end{document}